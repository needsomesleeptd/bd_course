\chapter*{\hfill{\centering  ЗАКЛЮЧЕНИЕ}\hfill}
\addcontentsline{toc}{chapter}{ЗАКЛЮЧЕНИЕ}

В результате исследования было определено что время обработки запроса к системе зависит от числа запросов секунду и наличия кеша. Кеш дает значительный выигрыш при сопоставимому числу объектов в нем и во всем множестве. При использовании кеша при 51000 обработанных объектах время обработки запроса было сокращено в 19.41 раз, что связано с более быстрому получнию данных из хранилища в оперативной памяти.

При исследовании зависимости времени обработки запроса от числа запросов в секунду было выявлено, что при увелечении числа запросов в секунду с 50 до 150 среднее время обработки увеличилось в 2.01 раз, что связано с конккуренцией пользователей за ресурсы системы.

Поставленная цель: разработка базы данных обогащения обучающей выборки для автоматизированной проверки отчета на соответствие нормативным требованиям была выполнена.

Для поставленной цели были выполнены все задачи:
\begin{itemize}
	\item проанализированы существующие решения;
	\item формализована задача и определен необходимый функционал;
	\item проанализированы способы хранения данных и системы управления базами данных, выбрана подходящая система для поставленной цели;
	\item спроектирована база данных, описаны ее сущности и связи;
	\item разработана база данных;
	\item происследована зависимость времени выполнения запроса от числа получаемых запросов в секунду;
	\item произведено сравнение скорости обработки запросов реализаций с кешем и без.
\end{itemize}
