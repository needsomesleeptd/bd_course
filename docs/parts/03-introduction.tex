\chapter*{ВВЕДЕНИЕ}
\addcontentsline{toc}{chapter}{ВВЕДЕНИЕ}

Во время обучения студентам регулярно приходится писать отчеты к различным видам работ (курсовые, лабораторные, научно-исследовательские работы и т. д.), при этом оформление работ должно соответствовать ГОСТ, что необходимо своевременно проверить и при необходимости отправить отчет на доработку, однако, количество студентов намного превышает количество нормоконтроллеров. Для ускорения процесса проверки возможно использование автоматических систем.

Целью курсовой работы является разработка базы данных обогащения обучающей выборки для автоматизированной проверки отчета на соответствие нормативным требованиям.

Для достижения цели научно-исследовательской работы требуется решить следующие задачи:
\begin{itemize}
	\item проанализировать существующие решения;
	\item формализовать задачу и определить необходимый функционал;
	\item проанализировать способы хранения данных и системы управления базами данных, выбрать подходящую систему для поставленной цели;
	\item спроектировать базу данных, описать ее сущности и связи;
	\item разработать базу данных;
	\item исследовать зависимость времени выполнения запроса от числа получаемых запросов в секунду;
	\item сравнить скорость обработки запросов реализаций с кешем и без.
\end{itemize}