
\chapter{Конструкторская часть}
В данной части работы будет описана реализация базы данных, описана ER диаграмма БД и принципиальная схема БД, также будут описаны схемы триггеров и хранимые процедуры.

\section{Формализация сущностей базы данных}

В соответствие с диаграммой~\ref{img:user_inter} были выделены следующие таблицы:
\begin{enumerate}
	\item student --- таблица студентов;
	\item normcontroller --- таблица нормоконтроллеров;
	\item mistakes --- таблица ошибок в отчетах студентов;
	\item mistake\_type~---~таблица типов ошибок (возможно введение новых типов ошибок во время работы системы);
	\item report --- отчет студента;
	\item feedback --- комментарий студента о ошибке (в случае ошибки преподавателя можно указать на это);
	\item reward --- награды для выдачи студентам, преуспевающим в выполнении лабораторных работ.
\end{enumerate}
На основе описанной информации была получена диаграмма сущность---связь, представленная на рисунке~\ref{img:ER_RU}.
\includeimage
{ER_RU} % Имя файла без расширения (файл должен быть расположен в директории inc/img/)
{f} % Обтекание (без обтекания)
{H} % Положение рисунка (см. figure из пакета float)
{1\textwidth} % Ширина рисунка
{Диаграмма сущность---связь} % Подпись рисунка

\section{Формализация взаимодействия с приложением}
Для описания сценариев взаимодействия пользователей с системой была создана диаграмма прецедентов, представленная на рисунке~\ref{img:BD_use_cases}.

\includeimage
{BD_use_cases} % Имя файла без расширения (файл должен быть расположен в директории inc/img/)
{f} % Обтекание (без обтекания)
{H} % Положение рисунка (см. figure из пакета float)
{1\textwidth} % Ширина рисунка
{Диаграмма прецедентов} % Подпись рисунка

Системе автоматической проверки необходимо проверить на правильность составные части отчета, что подразумевает детекцию рисунков, графиков, схем алгоритмов, списка используемых источников, а также формул.

На рисунках \ref{img:detection}--\ref{img:detection_inside} представлены результаты разработки системы детекции составных частей.

\includeimage
{detection} % Имя файла без расширения (файл должен быть расположен в директории inc/img/)
{f} % Обтекание (без обтекания)
{H} % Положение рисунка (см. figure из пакета float)
{1\textwidth} % Ширина рисунка
{IDEF0 обнаружение составных частей отчет 0 уровня} % Подпись рисунка

\includeimage
{detection_inside} % Имя файла без расширения (файл должен быть расположен в директории inc/img/)
{f} % Обтекание (без обтекания)
{H} % Положение рисунка (см. figure из пакета float)
{1\textwidth} % Ширина рисунка
{IDEF0 обнаружение составных частей отчета 1 уровня} % Подпись рисунка

После решения задачи детекции необходимо проверить составные части на соответствие ГОСТ 7.32, однако, финальный вердикт должен выноситься экспертом. На рисунках \ref{img:errors}--\ref{img:errors_inside} представлены результаты разработки данной системы.

\includeimage
{errors} % Имя файла без расширения (файл должен быть расположен в директории inc/img/)
{f} % Обтекание (без обтекания)
{H} % Положение рисунка (см. figure из пакета float)
{1\textwidth} % Ширина рисунка
{IDEF0 проверки составных частей отчета на соответствие ГОСТ 7.32 0 уровня} % Подпись рисунка

\includeimage
{errors_inside} % Имя файла без расширения (файл должен быть расположен в директории inc/img/)
{f} % Обтекание (без обтекания)
{H} % Положение рисунка (см. figure из пакета float)
{1\textwidth} % Ширина рисунка
{IDEF0 проверки составных частей отчета на соответствие ГОСТ 7.32 1 уровня} % Подпись рисунка






