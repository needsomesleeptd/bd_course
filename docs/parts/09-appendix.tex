\begin{appendices}
	\chapter{}
	\includeimage
	{bd_pres-1} % Имя файла без расширения (файл должен быть расположен в директории inc/img/)
	{f} % Обтекание (без обтекания)
	{H} % Положение рисунка (см. figure из пакета float)
	{1\textwidth} % Ширина рисунка
	{Титульный лист (слайд 1)} % Подпись рисунка
	
	\includeimage
	{bd_pres-2} % Имя файла без расширения (файл должен быть расположен в директории inc/img/)
	{f} % Обтекание (без обтекания)
	{H} % Положение рисунка (см. figure из пакета float)
	{1\textwidth} % Ширина рисунка
	{Цели и задачи работы (слайд 2)} % Подпись рисунка
	
	\includeimage
	{bd_pres-3} % Имя файла без расширения (файл должен быть расположен в директории inc/img/)
	{f} % Обтекание (без обтекания)
	{H} % Положение рисунка (см. figure из пакета float)
	{1\textwidth} % Ширина рисунка
	{Анализ существующих средств автоматизации (слайд 3)} % Подпись рисунка
	
	\includeimage
	{bd_pres-4} % Имя файла без расширения (файл должен быть расположен в директории inc/img/)
	{f} % Обтекание (без обтекания)
	{H} % Положение рисунка (см. figure из пакета float)
	{1\textwidth} % Ширина рисунка
	{Выделение ключевых фрагментов отчета (слайд 4)} % Подпись рисунка
	
	\includeimage
	{bd_pres-5} % Имя файла без расширения (файл должен быть расположен в директории inc/img/)
	{f} % Обтекание (без обтекания)
	{H} % Положение рисунка (см. figure из пакета float)
	{1\textwidth} % Ширина рисунка
	{Основные фрагменты отчетов с ошибками 1 (слайд 5)} % Подпись рисунка
	
	\includeimage
	{bd_pres-6} % Имя файла без расширения (файл должен быть расположен в директории inc/img/)
	{f} % Обтекание (без обтекания)
	{H} % Положение рисунка (см. figure из пакета float)
	{1\textwidth} % Ширина рисунка
	{Основные фрагменты отчетов с ошибками 2 (слайд 6)} % Подпись рисунка
	
	\includeimage
	{bd_pres-7} % Имя файла без расширения (файл должен быть расположен в директории inc/img/)
	{f} % Обтекание (без обтекания)
	{H} % Положение рисунка (см. figure из пакета float)
	{1\textwidth} % Ширина рисунка
	{Виды пользователей приложения (слайд 7)} % Подпись рисунка
	
	\includeimage
	{bd_pres-8} % Имя файла без расширения (файл должен быть расположен в директории inc/img/)
	{f} % Обтекание (без обтекания)
	{H} % Положение рисунка (см. figure из пакета float)
	{1\textwidth} % Ширина рисунка
	{Процесс проверки работ (слайд 8)} % Подпись рисунка
	
		\includeimage
	{bd_pres-9} % Имя файла без расширения (файл должен быть расположен в директории inc/img/)
	{f} % Обтекание (без обтекания)
	{H} % Положение рисунка (см. figure из пакета float)
	{1\textwidth} % Ширина рисунка
	{Сущности предметной области (слайд 9)} % Подпись рисунка
	
		\includeimage
	{bd_pres-10} % Имя файла без расширения (файл должен быть расположен в директории inc/img/)
	{f} % Обтекание (без обтекания)
	{H} % Положение рисунка (см. figure из пакета float)
	{1\textwidth} % Ширина рисунка
	{Выбор СУБД (слайд 10)} % Подпись рисунка
	
		\includeimage
	{bd_pres-11} % Имя файла без расширения (файл должен быть расположен в директории inc/img/)
	{f} % Обтекание (без обтекания)
	{H} % Положение рисунка (см. figure из пакета float)
	{1\textwidth} % Ширина рисунка
	{Схема ПО (слайд 11)} % Подпись рисунка
	
		\includeimage
	{bd_pres-12} % Имя файла без расширения (файл должен быть расположен в директории inc/img/)
	{f} % Обтекание (без обтекания)
	{H} % Положение рисунка (см. figure из пакета float)
	{1\textwidth} % Ширина рисунка
	{схема БД (слайд 12)} % Подпись рисунка
	
		\includeimage
	{bd_pres-13} % Имя файла без расширения (файл должен быть расположен в директории inc/img/)
	{f} % Обтекание (без обтекания)
	{H} % Положение рисунка (см. figure из пакета float)
	{1\textwidth} % Ширина рисунка
	{Результат проверки файлов (слайд 13)} % Подпись рисунка
	
		\includeimage
	{bd_pres-14} % Имя файла без расширения (файл должен быть расположен в директории inc/img/)
	{f} % Обтекание (без обтекания)
	{H} % Положение рисунка (см. figure из пакета float)
	{1\textwidth} % Ширина рисунка
	{Триггеры и хранимые процедуры (слайд 14)} % Подпись рисунка
	
		\includeimage
	{bd_pres-15} % Имя файла без расширения (файл должен быть расположен в директории inc/img/)
	{f} % Обтекание (без обтекания)
	{H} % Положение рисунка (см. figure из пакета float)
	{1\textwidth} % Ширина рисунка
	{Зависимость времени обработки запроса от числа запросов в секунду (слайд 15)} % Подпись рисунка
	
		\includeimage
	{bd_pres-16} % Имя файла без расширения (файл должен быть расположен в директории inc/img/)
	{f} % Обтекание (без обтекания)
	{H} % Положение рисунка (см. figure из пакета float)
	{1\textwidth} % Ширина рисунка
	{Зависимость времени обработки запроса от наличия/отсутствия кеша (слайд 16)} % Подпись рисунка
	
		\includeimage
	{bd_pres-17} % Имя файла без расширения (файл должен быть расположен в директории inc/img/)
	{f} % Обтекание (без обтекания)
	{H} % Положение рисунка (см. figure из пакета float)
	{1\textwidth} % Ширина рисунка
	{Заключение (слайд 17)} % Подпись рисунка
	
		\includeimage
	{bd_pres-18} % Имя файла без расширения (файл должен быть расположен в директории inc/img/)
	{f} % Обтекание (без обтекания)
	{H} % Положение рисунка (см. figure из пакета float)
	{1\textwidth} % Ширина рисунка
	{Направления дальнейшего развития (слайд 18)} % Подпись рисунка
	
	
	
	
	
	
	
	
	
	
	
	
	
	\chapter{}
	\label{app:Mist}
	\section{Основные ошибки в отчетах}
	В данном разделе будут рассмотрены наиболее часто встречающиеся ошибки, которые совершают студенты при написании различных отчетов.
	
	В целях выявления наиболее часто встречающихся ошибок были опрошены преподаватели, работа которых непосредственно связана с проверкой отчетов студентов.
	
	\subsection{Общие ошибки}
	В ГОСТ 7.32 указаны следующие размеры полей: левое --- 30 мм, правое --- 15 мм, верхнее и нижнее --- 20 мм~\cite{GOST732}. Выход за границы листа является одной из самых распространенных ошибок.
	
	Каждый объект (таблица, рисунок, схема алгоритма, формула) должен быть подписан и пронумерован, однако более подробно подписи к каждому из них будут рассмотрены в следующих подразделах.
	
	Если таблицу или схему не удается разместить на одной странице, то следует разбить данный объект на несколько частей, каждая из которых должна быть подписана.
	
	\subsection{Ошибки в тексте}
	Слова в тексте должны быть согласованы в роде, числе и падеже.
	
	Страницы отчета должны быть пронумерованы, однако, номер на титульном листе не ставится, но он является первой страницей, что означает, что следующая страница должна иметь номер ${2}$.
	
	Ненумерованный заголовок (введение, список литературы, оглавление и т. п.) должен быть выравнен по центру, при этом он состоит только из прописных букв (пример представлен в приложении~\ref{img:chapterNameMist}), другие варианты оформления являются не соответствующими стандарту.
	
	Абзацный отступ должен быть одинаковым по всему тексту отчета и равен 1,25 см~\cite{GOST732}. Любые другие варианты оформления считаются ошибочными.
	
	Возможна потеря научного стиля и переход к публицистике, что является ошибкой, текст работы должен быть написан на государственном языке в научном стиле.
	
	\subsection{Ошибки в рисунках}
	Частой ошибкой является неправильное оформление рисунков. Каждый рисунок должен быть подписан, при этом подпись должна располагаться строго по центру, внизу рисунка. Другое оформление считается ошибочным.
	
	Использование рисунков низкого разрешения является ошибкой. Все рисунки должны быть выполнены в высоком качестве, если обратное не требуется в самой работе.
	
	Некорректный поворот рисунка считается ошибкой. Если рисунок не удается разместить на странице, то допускается повернуть его таким образом, чтобы верх рисунка был ближе к левой части страницы (см. рисунок~\ref{img:imgRotateMist}).
	
	\subsubsection{Ошибки в графиках}
	Для каждого графика должна существовать легенда, для оформления которой существует два варианта:
	\begin{itemize}
		\item в одном из углов графика находится область, в которой указаны все обозначения;
		\item в подписи к графику описано каждое обозначение;
	\end{itemize}
	другое оформление является ошибкой.
	
	Часто на графиках отсутствуют единицы измерения, что является ошибкой. Должны быть подписаны единицы измерения каждой из осей графика, даже в том случае, если на графике оси подписываются словами, например, если измерение идет в штуках или на оси обозначены времена года (см. рисунок~\ref{img:graphAxesMist}).
	
	Отчеты могут быть напечатаны в черно-белом варианте, поэтому на графиках должны быть маркеры, которые позволят отличить графики друг от друга даже не в цветом варианте. Отсутствие маркеров считается ошибкой.
	
	При большом количестве графиков на одном рисунке возможна ситуация, при которой невозможно отличить один график от другого, что является ошибкой.
	
	\subsubsection{Ошибки в схемах алгоритмов}
	Если схему не удается разместить на одной странице, то она разбивается на несколько частей, каждая из которых должна быть подписана. Для разделения схемы алгоритма на части используется специальный символ-соединитель, который отображает выход в часть схемы и вход из другой части этой схемы, соответствующие символы-соединители должны содержать одно и то же уникальное обозначение, любые другие варианты оформления являются ошибочными.
	
	Часто вместо символа начала или конца алгоритма используют овал, однако в этом случае должен быть использован прямоугольник с закругленными углами (см. рисунок~\ref{img:schemeStartMist}).
	
	При использовании символа процесса (прямоугольник) часто используют прямоугольник с закругленными углами (см. рисунок~\ref{img:schemeRectMist}), что является ошибкой.
	
	При соединении символов схемы алгоритмов не нужны стрелки, если они соединяют символы в направлении слево-направо или сверху-вниз, в остальных случаях символы должны соединяться линиями со стрелкой на конце, отсутствие требуемых стрелок считается ошибкой.
	
	При использовании символа процесса-решение как минимум одна из соединительных линий должна быть подписана (см. рисунок~\ref{img:schemeDecMist}), однако возможен также вариант, когда подписаны обе линии. Отсутствие пояснений к выходам данного символа является ошибкой.
	
	Часто пояснительный текст пересекается с символами, использующимися для составления схем, что является ошибкой.
	
	\subsection{Ошибки в таблицах}
	Каждая таблица должна быть подписана. Наименование следует помещать над таблицей слева, без абзацного отступа в следующем формате: Таблица Номер таблицы - Наименование таблицы. Наименование таблицы приводят с прописной буквы без точки в конце~\cite{GOST732}. Другие варианты оформления считаются не соответствующими стандарту.
	
	Таблицу с большим количеством строк допускается переносить на другую страницу. При переносе части таблицы на другую страницу слово «Таблица», ее номер и наименование указывают один раз слева над первой частью таблицы, а над другими частями также слева пишут слова «Продолжение таблицы» и указывают номер таблицы~\cite{GOST732}. Любое другое оформление считается ошибочным.
	
	\subsection{Ошибки в формулах}
	Каждая формула должна быть пронумерована вне зависимости от того, существует ли ссылка на нее. Нумерация может осуществляться в двух вариантах:
	\begin{itemize}
		\item сквозная нумерация (номер формулы не зависит от раздела, в котором она находится);
		\item нумерация, зависящая от раздела (в том случае номер формулы начинается с номера раздела);
	\end{itemize}
	другое оформление считается ошибкой.
	
	Отсутствие знака препинания после формулы является ошибкой. После каждой формулы должен находиться знак препинания (точка, запятая и т.~п.), зависящий от контекста. Если в формуле содержится система уравнений, то после каждого из них (за исключением последнего) ставится запятая, а после последнего --- точка, либо запятая (см. рисунок~\ref{img:eqSystemMist}).
	
	Номер формулы должен быть выравнен по правому краю страницы и находиться по центру формулы (в вертикальной плоскости). Другое оформление нумерации формул считается не соответствующим стандарту.
	
	Если формула вставляется в начале страницы, то часто перед ней может присутствовать отступ, которого быть не должно.
	
	\subsection{Ошибки в списках}
	Ненумерованные списки должны начинаться с удлиненного тире (см. рисунок~\ref{img:itemizeMist}), другое оформление является ошибочным.
	
	В нумерованных списках после номера пункта обязательно должна стоять скобка (см. рисунок \ref{img:enumerateMist}), использование другого знака считается ошибкой.
	
	В конце каждого пункта списка должен быть знак препинания, от которого зависит первая буква первого слова следующего пункта (см. рисунок~\ref{img:itemizeLettersMist}):
	\begin{itemize}
		\item если пункт заканчивается на точку, то первое слово следующего пункта должно начинаться на прописную букву;
		\item если пункт заканчивается запятой или точкой с запятой, то следующий первое слово следующего слово должно начинаться со строчной буквы;
	\end{itemize}
	другое оформление является ошибочным.
	
	\subsection{Ошибки в списке литературы}
	Часто при описании одного из источников не указывается одна из составных частей (автор, издательство и т.~п.), что является ошибкой.
	
	Также нередко встречаются ссылки на так называемые <<препринтовские>> издательства (статья еще не вышла), однако была использована в отчете, это считается ошибкой.
	
	\chapter{}
	\includeimage
	{chapterNameMist} % Имя файла без расширения (файл должен быть расположен в директории inc/img/)
	{f} % Обтекание (без обтекания)
	{H} % Положение рисунка (см. figure из пакета float)
	{1\textwidth} % Ширина рисунка
	{Пример ошибочного оформления ненумерованного заголовка} % Подпись рисунка
	
	\includeimage
	{imgRotateMist} % Имя файла без расширения (файл должен быть расположен в директории inc/img/)
	{f} % Обтекание (без обтекания)
	{H} % Положение рисунка (см. figure из пакета float)
	{0.70\textwidth} % Ширина рисунка
	{Пример ошибочного оформления рисунка --- некорректный поворот} % Подпись рисунка
	
	\includeimage
	{graphAxesMist} % Имя файла без расширения (файл должен быть расположен в директории inc/img/)
	{f} % Обтекание (без обтекания)
	{H} % Положение рисунка (см. figure из пакета float)
	{1\textwidth} % Ширина рисунка
	{Пример ошибочного оформления графика --- отсутствуют единицы измерения} % Подпись рисунка
	
	\includeimage
	{schemeStartMist} % Имя файла без расширения (файл должен быть расположен в директории inc/img/)
	{f} % Обтекание (без обтекания)
	{H} % Положение рисунка (см. figure из пакета float)
	{0.5\textwidth} % Ширина рисунка
	{Пример ошибочного оформления схемы --- некорректный символ начала} % Подпись рисунка
	
	\includeimage
	{schemeRectMist} % Имя файла без расширения (файл должен быть расположен в директории inc/img/)
	{f} % Обтекание (без обтекания)
	{H} % Положение рисунка (см. figure из пакета float)
	{0.5\textwidth} % Ширина рисунка
	{Пример ошибочного оформления схемы --- некорректный символ процесса} % Подпись рисунка
	
	\includeimage
	{schemeDecMist} % Имя файла без расширения (файл должен быть расположен в директории inc/img/)
	{f} % Обтекание (без обтекания)
	{H} % Положение рисунка (см. figure из пакета float)
	{0.7\textwidth} % Ширина рисунка
	{Пример ошибочного оформления схемы --- не подписана ни одна из веток символа процесса---решение} % Подпись рисунка
	
	\includeimage
	{eqSystemMist} % Имя файла без расширения (файл должен быть расположен в директории inc/img/)
	{f} % Обтекание (без обтекания)
	{H} % Положение рисунка (см. figure из пакета float)
	{1\textwidth} % Ширина рисунка
	{Пример ошибочного оформления системы уравнений~---~отсутствуют знаки препинания после уравнений} % Подпись рисунка
	
	\includeimage
	{itemizeMist} % Имя файла без расширения (файл должен быть расположен в директории inc/img/)
	{f} % Обтекание (без обтекания)
	{H} % Положение рисунка (см. figure из пакета float)
	{1\textwidth} % Ширина рисунка
	{Пример ошибочного оформления ненумерованного списка~---~некорректный символ перед элементами списка} % Подпись рисунка
	
	\includeimage
	{itemizeLettersMist} % Имя файла без расширения (файл должен быть расположен в директории inc/img/)
	{f} % Обтекание (без обтекания)
	{H} % Положение рисунка (см. figure из пакета float)
	{1\textwidth} % Ширина рисунка
	{Пример ошибочного оформления ненумерованного списка~---~некорректный регистр буквы следующего пункта после запятой в предыдущем} % Подпись рисунка
	
	\includeimage
	{enumerateMist} % Имя файла без расширения (файл должен быть расположен в директории inc/img/)
	{f} % Обтекание (без обтекания)
	{H} % Положение рисунка (см. figure из пакета float)
	{1\textwidth} % Ширина рисунка
	{Пример ошибочного оформления нумерованного списка~---~некорректный символ после номера элемента списка} % Подпись рисунка
	
	
	\chapter{}
	\includelistingpretty
	{roles.sql} % Имя файла с расширением (файл должен быть расположен в директории inc/lst/)
	{sql} % Язык программирования (необязательный аргумент)
	{Реализация ролевой модели} % Подпись листинга
	
	\includelistingpretty
	{create.sql} % Имя файла с расширением (файл должен быть расположен в директории inc/lst/)
	{sql} % Язык программирования (необязательный аргумент)
	{Реализация создания отношений} % Подпись листинга
	
	\includelistingpretty
	{functions.sql} % Имя файла с расширением (файл должен быть расположен в директории inc/lst/)
	{sql} % Язык программирования (необязательный аргумент)
	{Реализация используемых функций} % Подпись листинга
	
	\includelistingpretty
	{triggers.sql} % Имя файла с расширением (файл должен быть расположен в директории inc/lst/)
	{sql} % Язык программирования (необязательный аргумент)
	{Реализация используемых триггеров} % Подпись листинга
	
	\chapter{}
	\includelistingpretty
	{measure_time.py} % Имя файла с расширением (файл должен быть расположен в директории inc/lst/)
	{sql} % Язык программирования (необязательный аргумент)
	{Реализация замера времени отработки запросов} % Подпись листинга
	
	
	
	
	
\end{appendices}
	
	
