
\chapter{Аналитическая часть}
В данной части работы будет описаны ошибки в отчетах, которые необходимо обнаружить, а также описаны участники процесса приема лабораторных работ, также будут описаны существующие средства автоматизации.

\section{Основные ошибки в отчетах}
Основные ошибки в отчетах описаны в приложении~\ref{app:Mist}.

\section{Прием лабораторных работ}
В не автоматизированной системе проверки отчетов на соответствие ГОСТ и дополнительным требованиям присутствуют две роли: студент, выполняющий некоторую работы, которая подразумевает написание отчета и нормоконтроллер, принимающий экспертное решение о соответствии предоставленного ему отчета необходимым требованиям.
\subsection{Участники процесса приема работ}
С помощью использования автоматической проверки отчета возможно
сократить временные ресурсы, выделяемые нормоконтроллером на проверку
отчетов студентов, однако, полностью отказаться от финального контроля
результатов человеком невозможно, таким образом существует две роли при
проверки отчета на соответствие ГОСТ, а именно: студент и нормоконтроллер.

\subsection{Процесс приема работ}

Студент отправляет отчет на проверку, а затем получает результат со
списком ошибок (если имеются). Нормоконтроллер же анализирует отчет,
составленный автоматической системой проверки, и при необходимости может
внести необходимые правки. Диаграммы процесса проверки отчетов приведены на картинках~\ref{img:ka}--\ref{img:stepsDiagram}, также приведена диаграмма BPMN~2.0, на которой представлено взаимодействие системы проверки отчетов, нормоконтроллера и студента~(рисунок \ref{img:user_inter}).

Использование автоматической проверки отчетов на соответствие ГОСТ и дополнительным требованиям сократит временные затраты на проверку отчетов.


\includeimage
{ka} % Имя файла без расширения (файл должен быть расположен в директории inc/img/)
{f} % Обтекание (без обтекания)
{H} % Положение рисунка (см. figure из пакета float)
{1\textwidth} % Ширина рисунка
{Диаграмма состояний проверки отчета} % Подпись рисунка

\includeimage
{stepsDiagram} % Имя файла без расширения (файл должен быть расположен в директории inc/img/)
{f} % Обтекание (без обтекания)
{H} % Положение рисунка (см. figure из пакета float)
{0.8\textwidth} % Ширина рисунка
{Диаграмма последовательности действий} % Подпись рисунка

\includeimage
{user_inter} % Имя файла без расширения (файл должен быть расположен в директории inc/img/)
{f} % Обтекание (без обтекания)
{H} % Положение рисунка (см. figure из пакета float)
{1\textwidth} % Ширина рисунка
{BPMN 2.0 диаграмма сдачи лабораторной работы} % Подпись рисунка


\section{Формализация требований к базе данных и приложению}
В ходе выполнения курсовой работы необходимо разработать базу данных для хранения информации о студентах их работ, достижениях и результатов проверки работ студентов, также стоит хранить разметку основных частей отчетов студентов.

Необходимо спроектировать и разработать приложение, которое позволит проверять отчеты студентов, сохранять их, а также сохранять новые разметки основных частей отчета.


\section{Анализ существующих средств автоматизации}
Ввиду распространенности решаемой проблемы уже были созданы приложения для автоматизации проверки документов на соответствие стандартам.

Наиболее популярными из них являются:
\begin{enumerate}
	\item ВКР-СМАРТ~\cite{VKR_VYZ};
	\item TestVkr~\cite{TestVkr};
	\item Applitools visual testing~\cite{PdfTest};
\end{enumerate}

Система ВКР-СМАРТ, предназначенная для проверки выпускных квалификационных работ (ВКР) студентов, представляет собой универсальную платформу, разработанную для системного хранения и проверки на заимствования ВКР и других работ обучающихся. Также система проверяет предложенные работы на выполнение всех требований ФГОС ВО и СПО, а также соответствие ГОСТам. После выполнения проверок, будет получен отчет о проценте заимствования и замечания по нарушенным стандартам~\cite{VKR_VYZ}.

Система ТЕСТ ВКР (Технический регламент проверки выпускных квалификационных работ) предназначена для проверки выпускных квалификационных работ студентов на объем заимствования и их размещения в электронно-библиотечной системе (ЭБС) университета. Система обеспечивает централизованное хранение и контроль за академическими работами студентов, а также их проверку на оригинальность и уникальность контента, также данную систему (без использования хранилища) возможно запускать локально, для проверки работы на нарушение ГОСТ~\cite{TestVkr}.

Платформа Applitools позволяет использовать <<визуальное тестирование>> предназначенное для сравнения получаемого изображения с реальным. Этот метод особенно эффективен при выявлении ошибок во внешнем виде страницы или экрана, которые могут остаться незамеченными при традиционном функциональном тестировании. С использованием Applitools Eyes разработчики могут легко интегрировать визуальные тесты, которые могут быть использованы для выявления отклонений от стандартов в PDF~\cite{PdfTest}.



\begin{table}[ht]
	\begin{center}
		\begin{threeparttable}
			\caption{\label{t:cmp} Сравнение существующих средств автоматизации}
			\begin{tabular}{|p{4cm}|p{4cm}|p{4cm}|c|}
				\hline
				\textbf{Критерий} & \textbf{ВКР СМАРТ} & \textbf{TestVkr} & \textbf{Applitools} \\ \hline
				Проверка текстов  & да & да & нет\\ \hline
				Проверка элементов отчета & нет & нет & да \\ \hline
				Наличие общего хранилища работ  & да & да & нет \\ \hline
				Возможность запуска локально & нет & *да & *да \\ \hline
			\end{tabular}
		\end{threeparttable}
	\end{center}
\end{table}


В таблице~\ref{t:cmp} под <<элементами отчета>> подразумеваются таблицы, рисунки, схема алгоритмов, формулы, использование символа *, означает, что в этом случае проверка плагиата не производится.

\section{Формализация информации, подлежащей хранению в проектируемой базе данных}
Разрабатываемая база данных, должна содержать информацию о следующих сущностях:
\begin{itemize}
	\item студент;
	\item нормоконтроллер;
	\item отчет студента;
	\item тип фрагмента отчета;
	\item фрагмент отчета;
	\item комментарий студента;
	\item достижение студента.
\end{itemize}

Сведения о каждой категории данных содержится в таблице~\ref{t:data_store}.

\begin{table}[ht]
	\begin{center}
		\begin{threeparttable}
			\caption{\label{t:data_store} Категории и сведения о данных}
			\begin{tabular}{|c|p{8cm}|}
				\hline
				\textbf{Категория} & \textbf{Сведения} \\ \hline
				Студент & ID источника данных, ID студента, никнейм, имя, фамилия, дата регистрации, число сданных лабораторных\\ \hline
				Отчет & ID отчета, номер попытки сдачи, файл отчета, время последнего обновления отчета, число проверок отчета, ID студента, значение того что отчет сдан \\ \hline
				Тип фрагмента отчета & ID фрагмента, описание типа фрагмента, ID контроллера, создавшего фрагмент отчета \\ \hline
				Выделенный фрагмент отчета & ID фрагмента отчета, изображение страницы отчета, данные фрагмента отчета (разметка), значение была ли ошибка проверена разметчиком, ID отчета, ID типа фрагмента отчета, ID проверяющего отчет\\ \hline
				Комментарий & ID ошибки, на которую создается комментарий, ID комментария, данные комментария, ID студента, создавшего комментарий \\ \hline
				Достижение & ID достижения, ID контроллера, создавшего достижение, данные достижения, описание достижения \\ \hline
			\end{tabular}
		\end{threeparttable}
	\end{center}
\end{table}

На основе описанной информации была получена диаграмма сущность---связь, представленная на рисунке~\ref{img:ER_RU}.
\includeimage
{ER_RU} % Имя файла без расширения (файл должен быть расположен в директории inc/img/)
{f} % Обтекание (без обтекания)
{H} % Положение рисунка (см. figure из пакета float)
{1\textwidth} % Ширина рисунка
{Диаграмма сущность---связь} % Подпись рисунка


\section{Анализ существующих баз данных}
База данных —-- это некоторый набор перманентных (постоянно хранимых) данных, используемых прикладными программными системами какого-либо предприятия~\cite{williams-db}.

Между физической базой данных (т.е. данными, которые реально хранятся на компьютере) и пользователями системы располагается уровень программного
обеспечения, который можно называть по-разному: диспетчер базы данных (database 
manager), сервер базы данных (database server) или, что более привычно, система управления базами данных, СУБД (DataBase Management System — DBMS).
Основная задача СУБД --- дать пользователю базы данных возможность работать с ней, не вникая во все подробности работы на уровне аппаратного обеспечения~\cite{williams-db}.

Модель данных — это абстрактное, самодостаточное, логическое определение объектов, операторов и прочих элементов, в совокупности составляющих абстрактную машину доступа к данным, с которой взаимодействует пользователь~\cite{williams-db}.
Рассмотрим классификацию баз данных по различным моделям данных:
\begin{enumerate}
	\item дореляционные;
	\item реляционные;
	\item постреляционные.
\end{enumerate}

\subsection{Дореляционные модели}
Дореляционные системы можно разделить на три большие категории:
\begin{enumerate}
	\item системы с инвертированными списками;
	\item иерархические;
	\item сетевые~\cite{williams-db}.
\end{enumerate}
Иерархическая база данных представляется множеством 
деревьев: в вершинах дерева помещаются записи, состоящие из поименованных 
полей и представляющие экземпляры некоторого объекта предметной области. 
Записи связаны строго иерархическими отношениями --- у записи-<<потомка>> не 
должно быть более одной записи-<<предка>>~\cite{wolf-db}.

Сетевая модель данных представляет собой логическую модель данных, которая расширяет иерархический подход. В иерархических структурах каждая запись-потомок имеет ровно одного предка, в то время как в сетевой структуре данных потомок может иметь несколько предков~\cite{wolf-db}.

Инвертированный список в общем случае — это двухуровневая индексная структура. На первом уровне находится файл или часть файла, в которой упорядоченно расположены значения вторичных ключей. Каждая запись с вторичным ключом имеет ссылку на номер первого блока в цепочке блоков, содержащих номера записей с данным значением вторичного ключа. На втором уровне находится цепочка блоков, содержащих номера записей, содержащих одно и то же значение вторичного ключа. При этом блоки второго уровня упорядочены по значениям вторичного ключа, на третьем уровне находится основной файл~\cite{inverted-lists}.

Механизм доступа к записям по вторичному ключу при подобной организации записей весьма прост. На первом шаге происходит поиск в области первого уровня заданное значение вторичного ключа, а затем по ссылке считываются блоки второго уровня, содержащие номера записей с заданным значением вторичного ключа, а далее уже прямым доступом загружается в рабочую область пользователя содержимое всех записей, содержащих заданное значение вторичного ключа~\cite{inverted-lists}.


\subsection{Реляционные модели}
Реляционная база данных — это такая база данных, которая воспринимается ее пользователями как множество переменных (т.е. переменных отношения), значениями которых являются отношения или, менее формально, таблицы~\cite{williams-db}.

Реляционная система, поддерживает реляционные базы данных и осуществляет операции над ними, включая RESTRICT (также известную как выборка или англ. SELECT), проекцию (англ. PROJECT) и соединение (англ. JOIN). Эти и подобные операции реляционной алгебры выполняются на уровне множеств~\cite{williams-db}.

\subsection{Постреляционные модели}
Современный (постреляционный) этап развития связан с использованием объектно-ориентированных технологий разработки программных систем и созданием СУБД нового поколения, унаследовавших все лучшее от дореляционных и реляционных систем. Постреляционные СУБД поддерживают 
объектные и объектно-реляционные модели данных и обеспечивают разработчикам возможность использовать объектно-ориентированные языки программирования, что дает таким системам технологические преимущества по сравнению с реляционными СУБД~\cite{wolf-db}.

\subsection{Хранение временных рядов}
Отдельно стоит отметить набирающие популярность СУБД временных рядов. Временные ряды (Time series) --- это данные, 
претерпевающие некоторые изменения с течением 
времени, и фиксируемые в конкретные промежутки 
времени. Данные СУБД оптимизированы под частую запись данных и скорость получения доступа к данным не настолько сильно зависит от числа хранимых данных, что характерно для реляционных баз данных, скорость работы которых уменьшается ввиду необходимости индексирования новых элементов. Однако во временных СУБД отсутствует механизм изменения записанных значений, так как считается что записанные метрики (данные) являются фактом в прошлом~\cite{time_db}.

\section*{Вывод}
В данном разделе были описаны основные ошибки студентов в отчетах по лабораторным работам, были выделены участники процесса приема лабораторных работ, формализован процесс приема лабораторных работ, а также рассмотрены существующие средства автоматизации проверки работ на соответствие стандартам.

