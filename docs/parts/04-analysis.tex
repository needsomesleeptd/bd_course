
\chapter{Аналитическая часть}
В данной части работы будет описаны ошибки в отчетах, которые необходимо обнаружить, а также описаны участники процесса приема лабораторных работ, также будут описаны существующие средства автоматизации.

\section{Основные ошибки в отчетах}
Основные ошибки в отчетах описаны в приложении~\ref{app:Mist}.

\section{Прием лабораторных работ}
В не автоматизированной системе проверки отчетов на соответствие ГОСТ и дополнительным требованиям присутствуют две роли: студент, выполняющий некоторую работы, которая подразумевает написание отчета и нормоконтроллер, принимающий экспертное решение о соответствии предоставленного ему отчета необходимым требованиям.
\subsection{Участники процесса приема работ}
С помощью использования автоматической проверки отчета возможно
сократить временные ресурсы, выделяемые нормоконтроллером на проверку
отчетов студентов, однако, полностью отказаться от финального контроля
результатов человеком невозможно, таким образом существует две роли при
проверки отчета на соответствие ГОСТ, а именно: студент и нормоконтроллер.

\subsection{Процесс приема работ}

Студент отправляет отчет на проверку, а затем получает результат со
списком ошибок (если имеются). Нормоконтроллер же анализирует отчет,
составленный автоматической системой проверки, и при необходимости может
внести необходимые правки. Диаграммы процесса проверки отчетов приведены на картинках~\ref{img:UseCase}--\ref{img:stepsDiagram}, также приведена диаграмма BPMN~2.0, на которой представлено взаимодействие системы проверки отчетов, нормоконтроллера и студента~(рисунок \ref{img:user_inter}).

Использование автоматической проверки отчетов на соответствие ГОСТ и дополнительным требованиям сократит временные затраты на проверку отчетов.
\includeimage
{UseCase} % Имя файла без расширения (файл должен быть расположен в директории inc/img/)
{f} % Обтекание (без обтекания)
{H} % Положение рисунка (см. figure из пакета float)
{0.9\textwidth} % Ширина рисунка
{Диаграмма вариантов автоматической проверки отчета} % Подпись рисунка

\includeimage
{ka} % Имя файла без расширения (файл должен быть расположен в директории inc/img/)
{f} % Обтекание (без обтекания)
{H} % Положение рисунка (см. figure из пакета float)
{1\textwidth} % Ширина рисунка
{Диаграмма состояний проверки отчета} % Подпись рисунка

\includeimage
{stepsDiagram} % Имя файла без расширения (файл должен быть расположен в директории inc/img/)
{f} % Обтекание (без обтекания)
{H} % Положение рисунка (см. figure из пакета float)
{0.8\textwidth} % Ширина рисунка
{Диаграмма последовательности действий} % Подпись рисунка

\includeimage
{user_inter} % Имя файла без расширения (файл должен быть расположен в директории inc/img/)
{f} % Обтекание (без обтекания)
{H} % Положение рисунка (см. figure из пакета float)
{1\textwidth} % Ширина рисунка
{BPMN 2.0 диаграмма сдачи лабораторной работы} % Подпись рисунка




\subsection{Анализ существующих средств автоматизации}
Ввиду распространенности решаемой проблемы уже были созданы приложения для автоматизации проверки документов на соответствие стандартам.

Наиболее популярными из них являются:
\begin{enumerate}
	\item ВКР-СМАРТ~\cite{VKR_VYZ};
	\item TestVkr~\cite{TestVkr};
	\item Applitools visual testing~\cite{PdfTest};
\end{enumerate}

Система ВКР-СМАРТ, предназначенная для проверки выпускных квалификационных работ (ВКР) студентов, представляет собой универсальную платформу, разработанную для системного хранения и проверки на заимствования ВКР и других работ обучающихся. Также система проверяет предложенные работы на выполнение всех требований ФГОС ВО и СПО, а также соответствие ГОСТам. После выполнения проверок, будет получен отчет о проценте заимствования и замечания по нарушенным стандартам~\cite{VKR_VYZ}.

Система ТЕСТ ВКР (Технический регламент проверки выпускных квалификационных работ) предназначена для проверки выпускных квалификационных работ студентов на объем заимствования и их размещения в электронно-библиотечной системе (ЭБС) университета. Система обеспечивает централизованное хранение и контроль за академическими работами студентов, а также их проверку на оригинальность и уникальность контента, также данную систему (без использования хранилища) возможно запускать локально, для проверки работы на нарушение ГОСТ~\cite{TestVkr}.

Платформа Applitools позволяет использовать <<визуальное тестирование>> предназначенное для сравнения получаемого изображения с реальным. Этот метод особенно эффективен при выявлении ошибок во внешнем виде страницы или экрана, которые могут остаться незамеченными при традиционном функциональном тестировании. С использованием Applitools Eyes разработчики могут легко интегрировать визуальные тесты, которые могут быть использованы для выявления отклонений от стандартов в PDF~\cite{PdfTest}.



\begin{table}[ht]
	\begin{center}
		\begin{threeparttable}
			\caption{\label{t:cmp} Сравнение существующих средств автоматизации}
			\begin{tabular}{|p{4cm}|p{4cm}|p{4cm}|c|}
				\hline
				\textbf{Критерий} & \textbf{ВКР СМАРТ} & \textbf{TestVkr} & \textbf{Applitools} \\ \hline
				Проверка текстов  & да & да & нет\\ \hline
				Проверка элементов отчета & нет & нет & да \\ \hline
				Наличие общего хранилища работ  & да & да & нет \\ \hline
				Возможность запуска локально & нет & *да & да \\ \hline
			\end{tabular}
		\end{threeparttable}
	\end{center}
\end{table}


В таблице~\ref{t:cmp} под <<элементами отчета>> подразумеваются таблицы, рисунки, схема алгоритмов, формулы, использование символа *, означает, что в этом случае проверка плагиата не производится.


\section*{Вывод}
В данном разделе были описаны основные ошибки студентов в отчетах по лабораторным работам, были выделены участники процесса приема лабораторных работ, формализован процесс приема лабораторных работ, а также рассмотрены существующие средства автоматизации проверки работ на соответствие стандартам.

